%%%%%%%%%%%%%%%%%%%%%%%%%%%%%%%%%%%%%%%%%%%%%%%%%%%%%%%%%%%%%%%%%%%%%%%
%%%%%%%%%%%%%%%%%%%%%%%%%%%%%%%%%%%%%%%%%%%%%%%%%%%%%%%%%%%%%%%%%%%%%%%
%%%%%                                                                 %
%%%%%     <file_name>.tex                                             %
%%%%%                                                                 %
%%%%% Author:      <author>                                           %
%%%%% Created:     <date>                                             %
%%%%% Description: <description>                                      %
%%%%%                                                                 %
%%%%%%%%%%%%%%%%%%%%%%%%%%%%%%%%%%%%%%%%%%%%%%%%%%%%%%%%%%%%%%%%%%%%%%%
%%%%%%%%%%%%%%%%%%%%%%%%%%%%%%%%%%%%%%%%%%%%%%%%%%%%%%%%%%%%%%%%%%%%%%%
\chapter{Conclusion}


\section{Conclusion}

    The goal of the project was to port Halide, an image processing language on \gls{hero}, and test the language on the hardware simulator.
    We first added Halide to the \gls{hero} repository and automated the build process with the already available tools, then we worked on porting Halide.
    The porting process was not straight forward as some modifications had to be done on the \gls{pulp} runtime.
    The compilation process of \gls{openmp} application had to be changed to enable Halide support.
    We then tested the implementation with a simple matrix multiplication kernel.
    We ran the same application on \gls{openmp} to compare the performance of both implementation on this particular workload.
    The results were on par with \gls{openmp}, and with some optimizations, it is possible to find a more optimized schedule easily.
    \Gls{openmp} will still outperform Halide in most of the workflow, as Halide was designed in the first place for image processing.

    During this project, Halide showed promising results, with its ease of programming and the performance of the code.
    A port of Halide on the full \gls{hero} platform would definitely be interesting for signal processing applications on the \gls{hero} platform.

\section{Future Work}

    \subsection{Cleaning}
        Currently, Halide is being ported on another branch and has not been tested, before merging the implementation with the main branch, further testing needs to be done to guarantee that the modified build of \gls{llvm} did not break any other toolchains.
        The branch requires in-depth testing before merging it to the main repository.

    \subsection{Porting Halide to the full platform}

    As an extension of the project, we tried to port Halide on the hardware platform.
    We tried two different approaches, one using the code exporting feature of Halide and the second one using the distribution instructions of \gls{openmp}.
    For the first method, we exported the pipeline into a C file and tried to import it in our application.
    This method didn't work as some part of the generated code use C++ and not C.
    Some headers are absent from the \gls{hero} SDK, so using this method require further investigation to determine which header are missing and why.
    The second method was to keep the pipeline as an object file, but adding a \texttt{  \#pragma omp target device(BIGPULP\_MEMCPY)} instruction before the pipeline call.
    Then the goal was to link the precompiled object file during compilation and create a heterogeneous application.
    But we did not have enough time to figure out how to compile heterogeneous applications using Halide.

