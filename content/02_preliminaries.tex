%%%%%%%%%%%%%%%%%%%%%%%%%%%%%%%%%%%%%%%%%%%%%%%%%%%%%%%%%%%%%%%%%%%%%%%
%%%%%%%%%%%%%%%%%%%%%%%%%%%%%%%%%%%%%%%%%%%%%%%%%%%%%%%%%%%%%%%%%%%%%%%
%%%%%                                                                 %
%%%%%     <file_name>.tex                                             %
%%%%%                                                                 %
%%%%% Author:      <author>                                           %
%%%%% Created:     <date>                                             %
%%%%% Description: <description>                                      %
%%%%%                                                                 %
%%%%%%%%%%%%%%%%%%%%%%%%%%%%%%%%%%%%%%%%%%%%%%%%%%%%%%%%%%%%%%%%%%%%%%%
%%%%%%%%%%%%%%%%%%%%%%%%%%%%%%%%%%%%%%%%%%%%%%%%%%%%%%%%%%%%%%%%%%%%%%%

\chapter{Preliminaries / Background}
\section{Hero}

\section{Halide Language}
	\subsection { Programing model}
		Halide is a functionnal program embedded into C++ designed to write high performance image and array processing code \cite{Web:Halide}. This language uses a functionnal paradigm to describe the functionnalities of the pipeline. The scheduling of the pipeline is described separately, which allow the developper to explore a wide range of schedule without having to rewrite most of the code. 
		Every pipeline is a function (\verb|Halide::Func| composed of other functions and expressions (\verb|Halide:expr|). These two objects  use special variables (\verb|Halide:Vars|) to describe the operation executed on the array. The code snippet 
		\ref{code:simple_pipeline} 
		describe a basic pipeline which compute the distance of each coordinate of the array from on position specified by the vector \verb|(center_x, center_y)|.


\lstset{basicstyle=\ttfamily\footnotesize,breaklines=true,tabsize=2}
\begin{lstlisting}[caption={Simple Pipeline Example}, captionpos=b, label={code:simple_pipeline}]
Halide::Var x, y;
Halide::Param center_x, center_y;
Halide::Expr offset = Halide::pow(x - center_x, 2) 
                      + Halide::pow(y - center_y, 2);

	gradient(x, y) = offset;
\end{lstlisting}

	After designing the  pipeline, we can define it's schedule via the different directive included in Halide. Halide implements all the basic scheduling option like parallelizing, unrolling, splitting ... These options will be described in the section 
	\nameref{section:scheduling}
	. The snippet 
	\ref{code:simple_pipeline_schedule} 
	shows a simple schedule applied on our gradient. This schedule consists of parallelizing the execution over the \texttt{x} axis, and unrolling along the  \texttt{y} axis.

	\begin{lstlisting}[caption={Simple Pipeline Example}, captionpos=b,label={code:simple_pipeline_schedule}];
	gradient.parallel(x);
	gradient.unroll(y, 10);
	\end{lstlisting}


	To execute our pipeline, Halide provides a large range of options, we can execute it directly using the \verb|.realize(x_max, y_max)| directive, this will execute the pipeline on a rectangle starting from it's top left corner in  \verb|(0,0)| to it's bottom right corner in \verb|(x_max, y_max)|.
	
	But Halide also gives the programmer a lot off options to execute the pipeline, it's able to convert the code to C code, llvm assembly file, or already compiled object file specific to a given target. More over, Halide support a wire variety of CPU architecture (X86, ARM, MIPS, PowerPc, Risc-V  ), operating systems ( Linux, Windows, Android, Mac) and also Gpu Api's (Cuda, OpenCL, OpenGl, DirectX ...). Halide support for new architecture is getting better and better, and is by design targeted for cross compilation and Heterogeneous systems.

	\subsection{Debugging Options}

	\subsection {Basic Scheduling Options}
	\label{section:scheduling}
	Halide implement different scheduling instruction, and most of them aren't architecture specific. 
	\subsubsection{Non Architecture Specific Instrutions}
	\begin{itemize}
		\item Reorder : Tells halide how to traverse the domain of the pipeline stage (ffor example in a column major or row major way)
		\item Split   : Split a dimension along inner and outer subdimensions.
		\item Tile    : Cut the domain in tiles.
		\item Fuse    : Join two dimensions in a single fused dimension.
		\item Unroll  : Unroll along one dimension.
	\end{itemize}

	Some of the primitives such as vectorize or parallel. needs to be implemented on the target platform as they take advantage of the specificities of the system. To do so Halide uses some functions which are defined in a header file we can get when compiling the pipeline.

	\subsection { Porting Halide to new Platforms}
		From the header file, we can find the functions vital for halide to work and implement them in the pulp runtime. From then, we have to compile the pipeline to a risc-V object file, and then compile the main application using this object file and the provided header. When we will compile the final application using the hero toolchain, the linker will link the halide function calls to the implemented functions in the pulp runtime.
		Currently, only the memory allocation, print, and fork primitive are implemented but they are sufficient to try some basic parallel schdeules.

\section{Compilation Workflow}
	Hero currently have different platforms, and also different workflow. I started by working on the simulation platform which simulate an eight-core pulp cluster. Then I tried to port it to the hardware platform (One hard ARM core and a PULP cluster implemented on the FPGA). The compilation is slightly different for both platform, so I started by working on the simulation platform which is not heterogeneous.
\section{Simulation}
	The compilation process for the simulation platform is quite easy,  we start by compiling the C++ code which will generate the final pipeline, we then run this application, to generate the pipeline and the matching header file.
	Then we compile the true application which will call the pipeline. During the process, we add the object file of the pipeline in the source file, g++ will then link the pipeline with the main application and also the halide calls to the pulp runtime functions.
		
\section{The full hero platform}
	The hardware platform has a more complex compiling process, currently, the applications use OpenMp to distribute the code to the pulp cluster. With instructions such as \verb| #pragma omp target device(BIGPULP_MEMCPY) | to  explicitely tell the compiler to distribute the following part of the code to the pulp cluster. The application is compiled using Clang and a the clang-offlad bundler to create an heterogeneous application that will run on both platform.

%%% Local Variables: 
%%% mode: latex
%%% TeX-master: "../report_template"
%%% End: 
