\hypertarget{introduction}{%
\subsection{Introduction}\label{introduction}}

Heterogeneous systems combine a general-purpose host processor with
domain-specific Programmable Many-Core Accelerators (PMCAs). Such
systems are highly versatile, due to their host processor capabilities,
while having high performance and energy efficiency through their PMCAs.
HERO is a FPGA-based research platform developed at IIS that combines a
PMCA composed by RISC-V cores, implemented as soft cores on an FPGA
fabric, with a hard ARM Cortex-A multicore host processor.

Heterogeneous systems have a complex programming model, which lead to
significant effort to develop tools to retain a high programmer
productivity. Halide is domain specific programming language designed to
write fast image processing algorithms. More specifically, it is a C++
dialect with a functional programming paradigm. It's aim is to separate
the function applied to the image (pipeline), and the sequence in which
the algorithm is executed (schedule). For example, the schedule
encompasses how the algorithm is parallelized, if the image is tiled,
processed in column or row major order, if solutions required by
multiple threads are shared or recomputed, if parts of the computation
is offloaded to an accelerator, and so on. This allows a programmer to
write a functional description of the image processing algorithm and
then explore ways of scheduling the execution with only a couple of
lines of code, and without modifying the algorithm. Furthermore, the
same algorithm can be run efficiently on multiple different
architectures by only changing the schedule. To have Halide generate
efficient code, the specific architecture requires to have an efficient
Halide runtime implementation, and good compiler support, as Halide is
tightly coupled with the compiler.

\hypertarget{project_description}{%
\subsection{Project description}\label{project_description}}

The goal of this project is to bring up Halide on HERO, using Ariane, a
64-bit RV64GC core, as a host processor. Ariane would manage Halide's
frontend, while the image processing tasks would execute on 32-bit cores
in the cluster. The final goal of this thesis is to have Halide
programmed image processing kernels running on an HERO system
implemented on an FPGA.

The project can be done by as one or two semester thesis. The project
consists of three parts:

1. Familiarizing with the Halide language and the architecture of HERO
(\textasciitilde2 person weeks).

2. Add a RISC-V target to Halide's frontend (\textasciitilde3 person
weeks).

3. Test up the Halide environment on an FPGA with a set of custom image
processing kernels (\textasciitilde1 person week)

4. Documentation and report writing (\textasciitilde1 person week)

\hypertarget{required_skills}{%
\subsection{Required skills}\label{required_skills}}

To work on this project, you will need:

\begin{itemize}
\item
  to have worked in the past with at least one RTL language
  (SystemVerilog or Verilog or VHDL). Having followed the VLSI 1 course
  is recommended.
\item
  to have prior knowlegde of the C++ programming language
\item
  to have prior knowledge of hardware design and computer architecture
\item
  to be motivated to work hard on a super cool open-source project
\end{itemize}

\hypertarget{status_in_progress}{%
\subsubsection{Status: In progress}\label{status_in_progress}}

\begin{itemize}
\item
  Student: Pierre-Hugues Blelly
\item
  Supervision: \href{:User:Matheusd}{Matheus Cavalcante},
  \href{:User:Sriedel}{Samuel Riedel}, \href{:User:Akurth}{ Andreas
  Kurth}
\end{itemize}

\hypertarget{professor}{%
\subsubsection{Professor}\label{professor}}

\begin{description}
\item[]
\href{http://www.iis.ee.ethz.ch/portrait/staff/lbenini.en.html}{Luca
Benini}
\end{description}

\hypertarget{meetings_presentations}{%
\subsection{Meetings \& Presentations}\label{meetings_presentations}}

The students and advisor(s) agree on weekly meetings to discuss all
relevant decisions and decide on how to proceed. Of course, additional
meetings can be organized to address urgent issues.

Around the middle of the project there is a design review, where senior
members of the lab review your work (bring all the relevant information,
such as prelim. specifications, block diagrams, synthesis reports,
testing strategy, ...) to make sure everything is on track and decide
whether further support is necessary. They also make the definite
decision on whether the chip is actually manufactured (no reason to
worry, if the project is on track) and whether more chip area, a
different package, ... is provided. For more details refer to
\href{http://eda.ee.ethz.ch/index.php/Design_review}{(1)}.

At the end of the project, you have to present/defend your work during a
15 min. presentation and 5 min. of discussion as part of the IIS
Colloquium.

\hypertarget{references}{%
\subsection{References}\label{references}}

\begin{enumerate}
\item
  Andreas Kurth, Pirmin Vogel, Alessandro Capotondi, Andrea Marongiu,
  Luca Benini. HERO: Heterogeneous Embedded Research Platform for
  Exploring RISC-V Manycore Accelerators on FPGA. CARRV' 2017.
  \href{https://doi.org/10.3929/ethz-b-000219249}{link}
\item
  Jonathan Ragan-Kelley, Andrew Adams, Sylvain Paris, Marc Levoy, Saman
  Amarasinghe, Frédo Durand. Decoupling Algorithms from Schedules for
  Easy Optimization of Image Processing Pipelines. SIGGRAPH 2012.
  \href{http://people.csail.mit.edu/jrk/halide12}{link}
\end{enumerate}
