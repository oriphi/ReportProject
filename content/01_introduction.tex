%%%%%%%%%%%%%%%%%%%%%%%%%%%%%%%%%%%%%%%%%%%%%%%%%%%%%%%%%%%%%%%%%%%%%%%
%%%%%%%%%%%%%%%%%%%%%%%%%%%%%%%%%%%%%%%%%%%%%%%%%%%%%%%%%%%%%%%%%%%%%%%
%%%%%                                                                 %
%%%%%     <file_name>.tex                                             %
%%%%%                                                                 %
%%%%% Author:      <author>                                           %
%%%%% Created:     <date>                                             %
%%%%% Description: <description>                                      %
%%%%%                                                                 %
%%%%%%%%%%%%%%%%%%%%%%%%%%%%%%%%%%%%%%%%%%%%%%%%%%%%%%%%%%%%%%%%%%%%%%%
%%%%%%%%%%%%%%%%%%%%%%%%%%%%%%%%%%%%%%%%%%%%%%%%%%%%%%%%%%%%%%%%%%%%%%%

\chapter{Introduction}

\section{Heterogeneous systems}

    Thanks to system integration and the better energy efficiency of modern CPUs, embedded systems are becoming quite powerful. They can now manage multiple sensors in real-time. But the power consumption is still an issue. These chips often have a limited power budget as their goal is to be embedded in small systems where the battery capacity doesn't exceed the hundreds of milli Ampere hour. 


    To improve the energy efficiency and the computing power of those chips, researcher have been trying to find new architecture which would suit those applications. Heterogeneous computing may be one of the solutions to this problem. Heterogeneous systems are often composed of one general-purpose CPU and multiple coprocessors which will be used to handle specific tasks \cite{Web:HeteroStrat}. This strategy has been used in the SoC industry by ARM since 2011 by ARM in it's big.LITTLE architecture \cite{Art:bigLITTlE}. This architecture based on two clusters of ARM Cortex A7 and A15, was designed to increase the computing power in embedded systems such as smartphones while increasing the battery life of the device. Even though this architecture relied on a single \acrshort{isa}, the idea was to use more powerful cores when needed and turn them off when the device was in sleep mode or when the lower power core could handle the task. 


	These architectures are interesting to increase the power efficiency but can also increase the processing power at the same time. Researchers have been trying to take advantage of multiple \acrshort{isa} and leverage their specificities to increase the overall performances with great success. In this article \cite{Art:Harnessing} from 2014, researchers compared homogeneous and heterogeneous systems based on three \acrshort{isa}, under strict design constraint (such as area or thermal dissipation) they observed that the heterogeneous designs performed better than their homogeneous counterpart. Heterogeneous systems are also heavily used in Data Centers, as the processing needs keep rising, they now use GPU as they are heavily optimized for parallel tasks and are now powerful enough to be used in scientific applications.

    Hero \cite{Web:Hero} is a heterogeneous system developed by the \acrfull{iis} and the \acrfull{eees}. This platform is composed of a hard multicore ARM 64 Juno \acrshort{soc} (composed of two Cortex A57 and four Cortex A553) and one PULP cluster (composed of eight RI5CY cores), running on an \acrshort{fpga}.

\section {Design Issue with heterogeneous systems}
	Due to their heterogeneous nature, those systems are difficult to program. The compilation process is complex, requiring multiple passes on different toolchains to distribute the work to the host and the accelerator, and to allocate the memory. And this complexity is also forced on the end-user, as he needs to know the target perfectly to make use of all the available resources.

\section { Currently Available Workflow for Halide}
	Currently Hero supports OpenMp, which is an API which "defines a portable, scalable model with a simple and flexible interface for developing parallel applications on platforms from the desktop to the supercomputer"\cite{Web:OpenMp}. This API has been implemented on hero, and is currently used to develop some test applications.
    But exploring the design space using OpenMp's directive, isn't perfect, and require the developer to adapt its code to run with a specific schedule. This approach leads to an important development time but also an extensive texting process whenever the schedule is changed to ensure that the resulting code works as intended.


    But exploring the design space using OpenMp's directive, isn't perfect, and require the developer to adapt its code to run with a specific schedule. This approach leads to an important development time but also an extensive texting process whenever the schedule is changed to ensure that the resulting code works as intended.

    Halide is a programming language that was designed to allow the developer to explore multiple design choices quickly by separating the algorithm from the execution schedule. This language was designed to be used in image or array processing applications.
    Every processing pipeline designed with Halide has two parts. The first part consist of the functional description of the processing kernel, this is the algorithm that will be executed on the array. And the second part is the schedule of the pipeline. This schedule describes how the algorithm will be executed on the system. This programming model is interesting because the developer can implement the algorithm without having to take into account the boundaries of the functions or the border effects. Then he can quickly bound the different variables of the pipeline and design it's schedule afterward. All the constraints will be asserted during the compilation without any intervention from the developer.


%%% Local Variables: 
%%% mode: latex
%%% TeX-master: "../report_template"
%%% End: 
