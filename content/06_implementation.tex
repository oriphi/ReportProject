%%%%%%%%%%%%%%%%%%%%%%%%%%%%%%%%%%%%%%%%%%%%%%%%%%%%%%%%%%%%%%%%%%%%%%%
%%%%%%%%%%%%%%%%%%%%%%%%%%%%%%%%%%%%%%%%%%%%%%%%%%%%%%%%%%%%%%%%%%%%%%%
%%%%%                                                                 %
%%%%%     <file_name>.tex                                             %
%%%%%                                                                 %
%%%%% Author:      <author>                                           %
%%%%% Created:     <date>                                             %
%%%%% Description: <description>                                      %
%%%%%                                                                 %
%%%%%%%%%%%%%%%%%%%%%%%%%%%%%%%%%%%%%%%%%%%%%%%%%%%%%%%%%%%%%%%%%%%%%%%
%%%%%%%%%%%%%%%%%%%%%%%%%%%%%%%%%%%%%%%%%%%%%%%%%%%%%%%%%%%%%%%%%%%%%%%



\chapter{Design Implementation}
    To test Halide, I used two applications. The first one was a basic gradient example~\ref{code:simple_pipeline}, and the second one a matrix multiplication pipeline that I took in the Halide repository and then adapted to be used in a hero application.
The matrix example is more interesting, as it represents what a typical signal processing application may do. It is also quite easy to benchmark with different sizes to see the impact of the memory access on the execution time.

\lstset{basicstyle=\ttfamily\footnotesize,breaklines=true,tabsize=2}
\begin{lstlisting}[caption={Matrix Multiplication Pipeline}, captionpos=b, label={code:matmul_pipeline}]
    ImageParam A(type_of<int>(), 2);
    ImageParam B(type_of<int>(), 2);
    Var x, y;
    Func matrix_mul("matrix_mul");
    Func out;

    RDom k( 0,A.width() );

    matrix_mul(x, y) += A(x, k) * B(k, y);

    out(x, y) = matrix_mul(x, y);

\end{lstlisting}
    The Listing~\ref{code:matmul_pipeline} shows the full algorithm implementation. The code is straight forward and is pretty close to the mathematical expression of the operation.


\section { Porting Halide to new Platforms}
    Halide is compiled using \gls{llvm}, as the \gls{hero} toolchain already has a build of this compiler, we can use it to compile Halide.
     Some of the build options are incompatible with Halide, the \texttt{-DBUILD\_SHARED\_LIBS} flag has to be disabled, and \gls{llvm} also needs to support the x86 \gls{isa} to compile Halide for the build computer.
    Halide can now be built automatically using the \texttt{tc-halide}  target in the main \texttt{Makefile}.
    Once we added Halide to the toolchains, we can work on porting it to \gls{hero}.


    The library header file generated by Halide explains which functions need to be implemented to make Halide work on our target platform. We can also use the error messages during the compilation to implement the missing functions.

    To make Halide work, we only need to port a small subset of functions. As most of the schedules are only reformatting the code.
    Only the parallel schedule has platform-specific code. So we can make Halide work with these functions, the memory allocation primitives, and the debugging functions (\texttt{halide\_printf}).

\section{Schedule Implementation }
    Most of the schedules implemented on halide does not require any platform-specific implementation as they are working with loops, but we have to add the missing functions to the \gls{pulp} runtime.

    \subsection{Modification to the \acrshort{pulp} runtime}

    The missing halide functions needs to be accessible to the \gls{pulp} runtime, to do so, we created a new file in the kernel (\texttt{halide\_api.c}). 
    This file contains all the basic functions required to run halide on \gls{hero}, and the \glspl{atomicOp}.

    Currently, only the \texttt{parallel()} instruction needs a specific function: \texttt{halide\_do\_par\_for}.
     This function initializes the \gls{pulp} cluster and add all the parallel tasks to the cluster queue. These task executes \texttt{halide\_do\_par\_for\_fork} which is a wrapper around the pipeline function.
    Each core select which task it will run based on the task id, if the id of the task modulo the number of cores is equal to the core id, the task will be executed.

    Halide does not support the RISC-V \gls{simd} extension, but the vectorize schedule may still be used, as Halide will reshape the code as if it was manipulating vectors.


 %%% WORK ON THE COMPILATION WORKFLOW

\section{Compilation Workflow}
    Every application has at least two source files, one C++ file which will generate the object file of the pipeline, the main application. 
    The compilation has two phases, during the first one, we compile the Halide application using \gls{llvm} and run it on the host platform, this application will then generate a RISC-V object file and a header.
    Then we compile the hero application using the same \texttt{Makefile} as the \gls{openmp} applications, but we also include the header in the main application and the object file to the sources during the linking command.

    \subsection{Compiling for the  full platform}
    This process only works on the hardware simulation, and I didn't achieved to make it work on the full \gls{hero} platform.
    I tried to approach the question using different strategy. The first one was to use the already compiled object file and add it during the linking process. This method didn't work as Clang didn't have any indication to distribute the code on the \gls{pulp} cluster. So the RISC-V object file was incompatible.

    The second idea was to use Halide to output C code and include the source in the \gls{hero} application. The issue is that the output of Halide is not pure C code, the pipeline function is coded in C but some structures are still using C++ style. The output needs to be modified by hand to be included in the application. But even after those modifications, the header creates incompatibilities, and thus this method is not usable.

    The last idea was to use the \gls{openmp} \texttt{\#pragma} call to distribute the execution of the function to the \gls{pulp} cluster, and then use the \gls{llvm} assembly file of the pipeline to include it in the application.
    As the first step of the compilation process for \gls{openmp} uses the \gls{llvm} assembly files to compute the offsets in memory, this method may be the best one to compile a Halide application for \gls{hero}.
    But I didn't have enough time to make the heterogeneous compilation work, so this method might not work.
