%%%%%%%%%%%%%%%%%%%%%%%%%%%%%%%%%%%%%%%%%%%%%%%%%%%%%%%%%%%%%%%%%%%%%%%
%%%%%%%%%%%%%%%%%%%%%%%%%%%%%%%%%%%%%%%%%%%%%%%%%%%%%%%%%%%%%%%%%%%%%%%
%%%%%                                                                 %
%%%%%     <file_name>.tex                                             %
%%%%%                                                                 %
%%%%% Author:      <author>                                           %
%%%%% Created:     <date>                                             %
%%%%% Description: <description>                                      %
%%%%%                                                                 %
%%%%%%%%%%%%%%%%%%%%%%%%%%%%%%%%%%%%%%%%%%%%%%%%%%%%%%%%%%%%%%%%%%%%%%%
%%%%%%%%%%%%%%%%%%%%%%%%%%%%%%%%%%%%%%%%%%%%%%%%%%%%%%%%%%%%%%%%%%%%%%%

\chapter{Background}


\subsection{Conclusion}
The goal of the project was to port Halide, an image processing language on \gls{hero}, and run image processing kernel on the test hardware.
Some functions were missing on the \gls{pulp} runtime, so we first added them to the source code. 
Then we added the Halide source to the \gls{hero} project and built the library. The compiling options of \gls{llvm} needed to be changed to sucessfully compile Halide.
Using the Makefile of the \gls{openmp} applications as a base, I sucessfully compiled Halide applications to run on the \gls{rtl} simulator.
I then tested two applications a gradient and a matrix multiplication to debug the schedules and test if they were working correctly.
After that, I ran some benchmarks on different matrix sizes to compare the perfromance of Halide and \gls{openmp} to determine whether Halide could compete with \gls{openmp} or not.
I then tried to make the heterogeneous compilation work on the \gls{hero} platform with the ARM host but I didn't have enough time make it work. I slightly changed the \gls{llvm} target to include other object file during linking. But in the end I didn't have enough time to make it work on the hardware platform.

Even if I couldn't finish the project, on the \gls{rtl} simulator, Halide showed promising results, but it need to be benchmarked more thoroughly to have a better idea of the performances Halide can achieve.

\subsection{Future Work}
A lot of work needs to be done to merge Halide on the main branch of \gls{hero}, The heterogeneous workflow for Halide needs to be fixed as it is  impossible right now to distribute code to the \gls{pulp} cluster from the ARM cluster. 
The \gls{ci} is currently not working, which is probably due to the change of options when compiling \gls{llvm}, this may also cause compability issues with \gls{openmp} or other components of the toolchain. This branch requires in depth testing before being merged with the main project.




