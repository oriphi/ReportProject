\documentclass[aspectratio=169]{beamer}
\usepackage[utf8]{inputenc}
\author{Pierre-Hugues BLELLY}
\date{08/06/2020}

\usepackage{listings}
\usepackage{graphicx}


\title{Implementation of a Heterogeneous System for Image Processing on an FPGA.}

\begin{document}
\lstset{basicstyle=\ttfamily\footnotesize,breaklines=true,tabsize=2}
\frame{\titlepage}

\begin{frame}{Introduction}
	\begin{itemize}
		\item Embedded systems, we need more performance as systems are becoming more complex
			(examples)
		\item Heterogenous Systems
			\begin{itemize}
				\item Everything is heterogeneous (Computers with network cards and HDD to supercomputer with GPGPU).
				\item Why do they exist?
					\begin{itemize}
						\item Better Energy efficiency under constraint
						\item Better flexibility thanks to PMCAs
					\end{itemize}
				\item Why do we need them?
					\begin{itemize}
						\item Autonomous systems
						\item Enable High processing power on ULP systems.
					\end{itemize}
			\end{itemize}
	\end{itemize}
\end{frame}

\begin{frame}{HERO Platform}
	\begin{itemize}
		\item What is HERO? Why does this system exist?
			\begin{itemize}
				\item Technical specs (configurations available: simulator + RISC/ Arm host + L1/L2 cache)
				\item Toolchains / SdK
				\item Goal of the project: Fully working platform, easier to experiment with heterogeneous systems.
			\end{itemize}
		\item Quick presentation of OpenMP, and it's advantages/disadvantages.
	\end{itemize}
\end{frame}

\begin{frame}{Halide}
	\begin{itemize}
		\item Presentation of Halide
			\begin{itemize}
				\item Programmation model
				\item Why this language is interesting for application developpement
			\end{itemize}
	\end{itemize}
\end{frame}

\begin{frame}[fragile]
	\frametitle{Simple Pipeline example}

% Changer l'exemple
	\begin{columns}[T]
		\begin{column}{.5\textwidth}
			\begin{lstlisting}[language=C]
ImageParam A(type_of<int>(), 2);
ImageParam B(type_of<int>(), 2);
Var x, y;
Func matrix_mul("matrix_mul");
Func out;
RDom k( 0,A.width() );
matrix_mul(x, y) +=
			A(x, k) * B(k, y);
out(x, y) = matrix_mul(x, y);

			\end{lstlisting}
		\end{column}
		\begin{column}{.5\textwidth}
			\begin{lstlisting}[language=C]
out.tile(x,y,x_o,y_o,x_i,y_i,4,4);
out.parallel(x_o);
out.parallel(y_o);
out.fuse(x_i,y_i,xy);
out.vectorize(xy);

			\end{lstlisting}
		\end{column}
	\end{columns}

	\begin{itemize}
		\item Pipeline definition: Functions, variables and expressions.
		\item Schedule definition: Basic Idea on how to schedule + description of some primitives
	\end{itemize}
\end{frame}

\begin{frame}{How to compile to the hardware simulation}
	\begin{itemize}
		\item Easiest way is to create a RISC-V object file.
		\item Add functions to the pulp-rt 
		\item Then we need to create a wrapper application that will be compiled using gcc
		\item And we link the object file during compilation
	\end{itemize}
\end{frame}

\begin{frame}{What about performances, does Halide perform well ?}
	\begin{itemize}
		\item Quick comparaison between Halide and OpenMP with the results in parallel and single threaded.
		\item good results on this application, but can we do better?
	\end{itemize}
\end{frame}

\begin{frame}{Optimization of a schedule}
	\begin{itemize}
		\item Explain methodology, plus benchmarks
		\item Automated tool for optimizing schedule
	\end{itemize}
\end{frame}

\begin{frame}{Next step: Compiling for the full Hero System}
	\begin{itemize}
		\item What we tried? 
			\begin{itemize}
				\item Using the risc V object file and linking it manually
				\item Using the OpenMP pragma calls to tell the compiler the function will run on PULP, then use the object file
				\item Output the code to C, and then include it
			\end{itemize}
	\end{itemize}
\end{frame}

\begin{frame}{Conclusion}
	\begin{itemize}
		\item What is working and what isn't
		\item Future work on Halide
	\end{itemize}
\end{frame}

\end{document}
