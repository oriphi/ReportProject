%%%%%%%%%%%%%%%%%%%%%%%%%%%%%%%%%%%%%%%%%%%%%%%%%%%%%%%%%%%%%%%%%%%%%%%
%%%%%%%%%%%%%%%%%%%%%%%%%%%%%%%%%%%%%%%%%%%%%%%%%%%%%%%%%%%%%%%%%%%%%%%
%%%%%                                                                 %
%%%%%     <file_name>.tex                                             %
%%%%%                                                                 %
%%%%% Author:      <author>                                           %
%%%%% Created:     <date>                                             %
%%%%% Description: <description>                                      %
%%%%%                                                                 %
%%%%%%%%%%%%%%%%%%%%%%%%%%%%%%%%%%%%%%%%%%%%%%%%%%%%%%%%%%%%%%%%%%%%%%%
%%%%%%%%%%%%%%%%%%%%%%%%%%%%%%%%%%%%%%%%%%%%%%%%%%%%%%%%%%%%%%%%%%%%%%%

\chapter{Results}
\section{Test Setup}
	I benchmarked two applications on two platforms. I benchmarked the halide port  on the hardware simulation for the PULP cluste, and one openMp matrix multiplication application on the developpement platform on a Xilinx ZCU102. For both application, I generated random matrices of différent sizes, and for each sizes multiplication I counted the number of cycles needed to complete the operation. 
	With this setup, we can easily compare the performances of halide and OpenMp in a real world scenario for at least two basic schedules: Single threaded and Multi Threaded.
	To give the results a more meaning I also calculated the number of operations per cycles where one operation can either be an addition a multiplication or a memory access (which take 2 cycles each), so for a matrix of size n, the number of operations to finish the multiplication is : $6 * n ** 3 + n ** 2$ (each coefficient needs $2n$ memory accesses, $2n$ additions and multiplications and one memory store).

\section{Comparaison between OpenMp and Halide on the different platforms}
